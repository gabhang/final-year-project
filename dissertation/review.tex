\chapter{Technology Review}

\section{CI/CD Pipeline}
CI/CD is a set of DevOps practices that automate the software delivery process \cite{saarenpaa2020creating}. It ensures that new functionality is pushed through an automated workflow, known as a pipeline, which runs software tests and quality assurance, before the code is deployed. The pipeline enables developers to deploy new code changes quickly and safely, with the ability to roll back at any stage \cite{bs}. To achieve CD, it is important to practice CI beforehand, as this step is essential for achieving the continuous delivery of software \cite{dsmgm}. This is because the pipeline ensures that code is tested and quality assured before being deployed, reducing the chances of errors and defects in production, which ensures that the software is reliable and performs as expected. 

One of the major advantages of the CI/CD pipeline is the immediate feedback that developers receive on their code \cite{bs}. This helps developers identify and fix bugs quickly, thus reducing the time and effort needed for fixing bugs. According to the 2022 State of DevOps Report \cite{clark}, CI and CD can increase code deployment efficiency by as much as 208 times. This shows that this approach is an important step for ensuring efficiency in software development. 

The CI/CD pipeline also promotes agile software development. By automating the deployment process, developers can quickly and easily deploy new code changes, reducing the time to market for new features, which then saves time and money \cite{bs, phillips2015manager}. The faster software is deployed, the more revenue/profit can be generated. In addition, the automated process reduces the need for manual intervention, which saves time and reduces the risk of errors. More than half of the respondents in a survey are currently using or are planning to use the CI/CD approach \cite{clark}. This clearly shows that more and more developers are looking into the advantages of CI/CD.

However, the CI/CD pipeline can be complex and takes time to learn \cite{sander}. Developers need to understand the process and tools involved, which can be daunting for beginners. Although the process is automated, some steps in the pipeline may require manual approval. This can slow down the deployment process and introduce delays which cuts down company revenue \cite{sb}. Some tests require human supervision, which can slow down the deployment process \cite{laster}. This is particularly true for tests that require visual inspection, such as user interface tests.

Furthermore, the CI/CD pipeline can introduce security risks if not properly implemented. Developers need to be aware of potential vulnerabilities and take steps to mitigate them. The introduction of CI/CD has not been flawless. For example, half of organisations doing so do not include any security testing elements, according to a 451 Research survey of 350 enterprise IT decision-makers in North America and Europe fielded in May 2018 \cite{clark}. This might be due to that the developers want to deliver the features faster then it should take. While speed is important, it is important to go as fast as possible but no faster. Going too fast can introduce errors and defects, which can be costly to fix \cite{clark}. 

To summarise that, the CI/CD pipeline is an essential DevOps practice that automates the software delivery process. It provides immediate feedback to developers, promotes efficiency and agility, and saves time and money. However, it can be complex and require a significant learning curve. CI/CD pipeline can encourage developers to go too fast before they are ready. Developers need to be aware of potential risks, such as security vulnerabilities and the need for human supervision in some tests. Ultimately, the CI/CD pipeline can help organizations deliver software faster and more reliably, but it requires a disciplined approach to implementation and management.