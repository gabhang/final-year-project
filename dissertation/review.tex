\chapter{Technology Review}

\section{Continuous Integration and Continuous Deployment (CI/CD)}
CI/CD is a set of DevOps practices that automate the software delivery process \cite{saarenpaa2020creating}. It ensures that new functionality is pushed through an automated workflow, known as a pipeline, which runs software tests and quality assurance, before the code is deployed. The pipeline enables developers to deploy new code changes in a short period of time, with the ability to roll back at any stage safely \cite{bs}. To achieve CD, it is important to practice CI beforehand, as this step is essential for achieving the continuous delivery of software \cite{dsmgm}. This is because the pipeline ensures that code is tested and quality assured prior to deployment, lowering the chances of errors and defects in production, which ensures that the software is reliable and performs as expected. 

One of the major advantages of the CI/CD pipeline is the instant feedback that developers get based on their code \cite{bs}. This helps developers identify and fix bugs quickly, thus reducing the time and effort needed for fixing bugs. According to the 2022 State of DevOps Report \cite{clark}, CI and CD can increase code deployment efficiency by as much as 208 times. This shows that this approach is an important step for ensuring efficiency in software development. 

The CI/CD pipeline also promotes agile software development and DevOps practices. By automating the deployment process, developers can quickly and easily deploy new code changes, reducing the time to market for new features, which then saves time and money \cite{bs, phillips2015manager}. The faster software is deployed, the more revenue/profit can be generated. In addition, the automated process reduces the need for manual intervention, which saves time and reduces the risk of errors. More than half of the respondents in a survey are currently using or are planning to use the CI/CD approach \cite{clark}. This clearly shows that more and more developers are looking into the advantages of CI/CD.

However, the CI/CD pipeline can be complex and takes time to learn \cite{sander}. Developers need to understand the process and tools involved, which can be daunting for beginners. Although the process is automated, some steps in the pipeline may require manual approval. This can slow down the deployment process and introduce delays which cuts down company revenue \cite{sb}. Some tests require human supervision, which can slow down the deployment process \cite{laster}. This is particularly true for tests that require visual inspection, such as user interface tests.

Furthermore, the CI/CD pipeline can introduce security risks if not properly implemented. Developers need to be aware of potential vulnerabilities and take steps to mitigate them. The introduction of CI/CD has not been flawless. For example, half of organisations doing so do not include any security testing elements, according to a 451 Research survey of 350 enterprise IT decision-makers in North America and Europe fielded in May 2018 \cite{clark}. This might be due to that the developers want to deliver the features faster then it should take. While speed is important, it is important to go as fast as possible but no faster. Going too fast can introduce errors and defects, which can be costly to fix. 

To summarise that, the CI/CD pipeline is an essential DevOps practice that automates the software delivery process. It provides immediate feedback to developers, promotes efficiency and agility, and saves time and money. However, it can be complex and require a significant learning curve. CI/CD pipeline can encourage developers to go too fast before they are ready. Developers need to be aware of potential risks, such as security vulnerabilities and the need for human supervision in some tests. Ultimately, the CI/CD pipeline can help organizations deliver software faster and more reliably, but it requires a disciplined approach to implementation and management.

\section{Agile Methodology}
Agile software development methodology is becoming increasingly popular as it aims to address the insufficiency in traditional software development processes. With over 90\% of companies utilizing an agile approach for software development, it seeks to ensure a close link between the customer and developers to ensure that software meets market needs while striving for a more rapid release schedule \cite{hlrf}. Table~\ref{tab:swmethod} from the previous section compares agile with traditional methods which proves why agile is preferable than traditional methods.

Agile methodology is characterised by its light nature, which makes it response to change quickly \cite{koch}. Survey studies by the Standish group have reported that less than 50\% of software development projects are successfully delivered within the set time, budget, and scope. As such, Agile technique ensures that each feature is continuously enhanced until it satisfies the ultimate user requirement \cite{aaa}. This is feasible as Agile methods emphasize continuous feedback and improvement, leading to quicker responses and faster delivery \cite{dragos}. Each release adds new features or functionalities, which improves the overall quality of the software. Agile also optimizes communication, favoring face-to-face communication, which enhances the overall quality of communication between the customer and developers\cite{koch}.

Although Agile methodology is effective in direct communication, it is considered weak in documentation, making it less suitable for long-term software maintenance \cite{aaa, koch}. The lack of emphasis on documentation can lead to different recollections of the same exchange, which can cause confusion and errors in the development process. Despite the advantages of increasing the frequency of software development, an issue has developed because the Operations function (Ops) and the Development function (Dev) are not aligned \cite{hlrf}. This is because Agile focuses on the Dev part but not the Ops part. This misalignment can cause long delays in software releases to customers, ultimately hindering the overall success of the project.

Overall, Agile is a suitable methodology for projects that demand flexibility, quick responses to change, and close collaboration between developers and customers. Conversely, long-term projects or maintenance is not recommended to use Agile as documentation is not prioritised which may cause problem in the long run. It is important to consider the nature of the software development project and choose the most suitable methodology accordingly. 

\section{DevOps Approach}
DevOps is the combination of development and operations practices that emphasize automation, collaboration, and monitoring throughout the software development lifecycle \cite{joakim}. DevOps is an extension of Agile methodology, incorporating operations practices to provide an end-to-end solution from design to monitoring, with a focus on automation and continuous feedback. DevOps emerged to address the deficiency that prevented development teams from delivering to operations in a faster and more frequent manner \cite{hlrf}. Many organizations have adopted DevOps practices to improve their software development processes as shown in various surveys and studies \cite{spj}.

DevOps' primary benefit is the little manual intervention required for speedy and smooth application deployment \cite{spj}. DevOps may improve performance and efficiency, making the software development process quicker, more dependable, and more efficient by automating development, build, test, and deployment. Also, the automated process eliminates manual labor, allowing team members to concentrate on other duties \cite{joakim}. Moreover, DevOps emphasizes collaboration between development and operations teams, enabling faster delivery speed, reliability, and scalability. DevOps also enhances operations by configuring, deploying, and monitoring applications, leading to better utilisation of resources and less time wasted \cite{os}. The continuous integration (CI) aspect of DevOps makes applications more robust and reliable. By continuously testing and integrating code, DevOps ensures that code is always production-ready. The automation aspect of DevOps also helps in identifying and fixing issues quickly, leading to improved software quality.

However, putting DevOps into practice has its share of difficulties. Working with DevOps implementations presents several difficulties, one of which is that it is simple to lose sight of the intended outcome \cite{joakim}. DevOps is about delivering increased business value, not just about doing things faster. Therefore, it is important to keep track of the goal and measure the outcomes to ensure that the DevOps implementation is delivering value. Another challenge with DevOps is that it requires a significant cultural shift in the organization. The cultural shift can be difficult to achieve, as it requires all team members to work together and take ownership of the process. Additionally, the automation aspect of DevOps requires expertise in the use of automation tools, which can be a challenge for some team members. Finally, DevOps can be expensive to implement, requiring the use of expensive automation tools and a dedicated team to manage the automation process. The initial investment may be high, and it may take some time to realize the benefits of DevOps.

In conclusion, DevOps is a software development methodology that has become increasingly popular in recent years due to its ability to enhance automation, collaboration, and monitoring throughout the software development lifecycle. DevOps has several advantages, such as minimal manual intervention, faster deployment, and improved reliability. However, there are also challenges associated with DevOps, including the need for a cultural shift, expertise in automation tools, and high implementation costs. Organizations must weigh the pros and cons of DevOps and determine if it is the right methodology for their software development needs.

\section{Jira}
Jira is a project management tool that has gained increasing popularity for software development teams, especially those that use agile methodologies. The issue tracking system in Jira is particularly useful for Scrum, which emphasize iterative and incremental development \cite{patrick2, ravis, davidh}. It allows teams to track their work, manage projects, and collaborate in a single platform. 

Jira is known for its user-friendly interface. The user interface is very straightforward and easy to understand which enables users to get the hang of it in a short period of time. This feature is particularly important for teams that need to get started quickly and do not have a lot of time for training or research \cite{patrick1}. Apart from its ease of use, Jira also provides extensive data reporting capabilities that enable teams to generate custom reports and visualize data to track progress and identify areas for improvement. Reports such as burnup chart burndown chart and cumulative flow diagram can be generated automatically by Jira,  Additionally, Jira offers a wide range of features that can enhance efficiency, such as a robust searching facility that enables users to search for issues and tasks using various criteria  such as keywords, assignees, and due dates \cite{patrick1, ravis}. This feature makes it easier for software development teams to find what they need quickly and efficiently. 

Despite its many advantages, Jira has some limitations, one of which is its cost, particularly for small teams or startups, as some of its features require payment. This cost can add up as the team size increases \cite{ravis}. However, Jira does offer a free plan that teams can choose to use. Therefore, development teams should carefully consider the costs associated with using Jira before deciding to adopt it for their projects.

In summary, Jira is an effective project management tool that provides numerous benefits for agile software development teams. Its user-friendly interface, extensive data reporting capabilities, and integration with other tools make it a valuable asset for teams looking to enhance their efficiency and collaboration. However, teams should carefully consider the costs associated with using Jira before deciding to adopt it for their projects.

\section{Heroku}
Heroku is a platform as a service (PaaS) that changes the way web applications are deployed by providing an easy-to-use platform that allows developers to deploy, manage, and scale their applications \cite{mike}. In 2012, Heroku won the InfoWorld Technology of the Year award, cementing its position as a leading platform as a service provider with focus on infrastructure \cite{greengard, news}.

One of the significant advantages of Heroku is its flexibility \cite{greengard, anubhav, patrick}. Heroku supports a wide range of programming languages, including many modern open-source languages such as GO and Ruby. The list of supported languages is still growing, which makes it an attractive option for developers who want to work with the latest tools and technologies. Additionally, Heroku is highly scalable, and it scales transparently as traffic spikes, making it capable of serving applications with over 10,000 sustained requests per second today \cite{anubhav}.

Heroku's user-friendly interface and ease of use are also an advantage \cite{greengard}. Developers can easily connect to Heroku from Git, GitHub, and Docker, and the platform's adaptability means that it can adapt itself to the needs and requirements of the software. This makes it a popular choice for developers who want to focus on coding and development, rather than infrastructure management.

However, there are some limitations to using Heroku. One of the significant disadvantages is that heavy projects do not always run well on the platform \cite{greengard}. Heroku is designed to be easy to use and manage, which means that it may not be the best option for large, complex applications that require high levels of customization and control. Additionally, Heroku's pricing structure can be relatively expensive compared to other cloud platforms, especially for projects that require a lot of resources.

Overall, Heroku is a popular platform as a service that provides developers with a flexible, scalable, and user-friendly way to deploy and manage their web applications. Its focus on infrastructure and support for modern open-source languages makes it a popular choice for developers. However, its limitations, such as poor performance for heavy projects and a relatively expensive pricing structure, may make it unsuitable for certain projects. Developers should consider the specific requirements of their project before deciding to use Heroku as their deployment platform.

\section{GitHub Actions}
GitHub Actions is an innovative CI/CD solution provided by GitHub. It is a cloud-based, online service that can help researchers track, organize, discuss, share, and collaborate on software and other materials related to research production, including data, code for analyses, and protocols \cite{ds, kimetal}. And...it is free!

GitHub Actions is a CI/CD solution that allows for the automation of various stages in software development, including build, test, and deployment. It enables developers to create workflows using YAML configuration files. One of the main benefits of using GitHub Actions is that it is easy to design pipelines for code that is already hosted on GitHub. This makes it easier to automate the entire software development process, from code changes to production deployment. Another advantage of GitHub Actions is that it can alert developers of failing processes through email \cite{kimetal}. This means that developers can be notified of any issues in real-time and can address them quickly. This can help reduce downtime and increase the reliability of the software development process.

However, one of the disadvantages of using GitHub Actions is that it requires familiarity with YAML configuration files. Developers who are new to YAML may need to spend some time learning and understanding the syntax \cite{ds}. This initial research and learning process can be challenging for some, but once developers become familiar with YAML, they can benefit from the power and flexibility that GitHub Actions provides.

All in all, GitHub Actions is a powerful tool that can help researchers streamline their software development process. It offers many benefits, including free usage, easy pipeline design, and real-time alerts for failing processes. Although there is a learning curve associated with YAML configuration files, the benefits of using GitHub Actions make it a worthwhile investment for researchers looking to improve their software development process.

\section{GO Language}
GO is an open-source programming language developed by Google that is popular among web services and web applications, including Docker and Netflix. The language was designed with the primary goal of creating a modern language that could solve real-world problems, making it easier for developers to write code with practical warning and error messages \cite{mihalis}. 

One of the significant advantages of GO is its rich library, which allows developers to leverage existing code to solve common problems and build complex applications more efficiently, saving time and effort. Additionally, GO is portable and supports every operating system fully, making it possible to run on any platform which is not feasible in other modern languages\cite{andrew}. GO is famous for its concurrency, procedural, and distributed programming. Goroutines in GO can handle up to several thousand threads at the same time, making it useful for building web applications which handle multiple transactions  \cite{andrew, cgptt}. Furthermore, GO features garbage collection, which frees developers from worrying about memory allocation or deallocation \cite{andrew}. Objects that are not needed anymore will be freed automatically.

However, GO also has some limitations, including slower processing speed compared to languages like C, and a lack of direct support for object-oriented programming, which can be challenging for developers accustomed to that programming style \cite{mihalis}. Additionally, while GO's standard library is rich in features, it may not have as many third-party libraries as other languages, which could limit the options available for developers when they want to use a specific library in their project. 

In conclusion, GO is a versatile language that prioritizes practicality and ease of use, making it popular among many developers. Its rich library, portability, and support for concurrency and distributed programming make it ideal for building web applications. However, developers must consider its limitations, such as slower processing speed and lack of direct support for object-oriented programming, before deciding to use GO for their project.

\section{MERN Stack}
MERN is a full JavaScript stack that comprises of four different technologies: MongoDB, Express, React, and Node.js. It is a popular stack that is widely used for developing web applications, especially for projects that require rapid prototyping.

One of the significant advantages of using MERN is that all four components use the same language, JavaScript. This allows developers to save time and increase productivity as they don't have to switch between different programming languages. Moreover, using a consistent language across the stack ensures that the codebase is uniform and easy to maintain \cite{shama}. Another benefit of MERN is that it provides better performance compared to the MEAN stack, which uses Angular. React, a component of MERN, is known for its high performance and efficient rendering. This makes MERN a great choice for building large-scale, data-intensive applications.Security is another advantage of using MERN. Since MERN uses JavaScript for both server-side and client-side development, it provides a secure environment for sensitive data. Moreover, MERN has an extremely low latency, which ensures that the application loads quickly and efficiently.

Despite its advantages, MERN has a steep learning curve, which can be challenging for new developers. It requires a solid understanding of JavaScript and its associated frameworks, as well as the ability to integrate them seamlessly \cite{eddy}. Another disadvantage of MERN is that it runs in the browser, not on the user's server \cite{asvj}. This can lead to slower performance if the user has a poor internet connection or if the application is not optimized for performance. However, this can be mitigated by using caching and optimizing the application for speed.

In conclusion, the MERN stack has several advantages, including a consistent language across all components, better performance compared to MEAN stack, and enhanced security. However, it also has a steep learning curve and may suffer from slow performance if the user has a poor connection. Overall, the MERN stack is a useful tool for developers who want to build fast and secure web applications, but it requires a certain level of expertise in JavaScript to be able to use it effectively.