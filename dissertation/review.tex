\chapter{Technology Review}

\section{CI/CD Pipeline}
CI/CD is a set of DevOps practices that automate the software delivery process \cite{saarenpaa2020creating}. It ensures that new functionality is pushed through an automated workflow, known as a pipeline, which runs software tests and quality assurance, before the code is deployed. The pipeline enables developers to deploy new code changes quickly and safely, with the ability to roll back at any stage \cite{bs}. To achieve CD, it is important to practice CI beforehand, as this step is essential for achieving the continuous delivery of software \cite{dsmgm}. This is because the pipeline ensures that code is tested and quality assured before being deployed, reducing the chances of errors and defects in production, which ensures that the software is reliable and performs as expected. 

One of the major advantages of the CI/CD pipeline is the immediate feedback that developers receive on their code \cite{bs}. This helps developers identify and fix bugs quickly, thus reducing the time and effort needed for fixing bugs. According to the 2022 State of DevOps Report \cite{clark}, CI and CD can increase code deployment efficiency by as much as 208 times. This shows that this approach is an important step for ensuring efficiency in software development. 

The CI/CD pipeline also promotes agile software development. By automating the deployment process, developers can quickly and easily deploy new code changes, reducing the time to market for new features, which then saves time and money \cite{bs, phillips2015manager}. The faster software is deployed, the more revenue/profit can be generated. In addition, the automated process reduces the need for manual intervention, which saves time and reduces the risk of errors. More than half of the respondents in a survey are currently using or are planning to use the CI/CD approach \cite{clark}. This clearly shows that more and more developers are looking into the advantages of CI/CD.

However, the CI/CD pipeline can be complex and takes time to learn \cite{sander}. Developers need to understand the process and tools involved, which can be daunting for beginners. Although the process is automated, some steps in the pipeline may require manual approval. This can slow down the deployment process and introduce delays which cuts down company revenue \cite{sb}. Some tests require human supervision, which can slow down the deployment process \cite{laster}. This is particularly true for tests that require visual inspection, such as user interface tests.

Furthermore, the CI/CD pipeline can introduce security risks if not properly implemented. Developers need to be aware of potential vulnerabilities and take steps to mitigate them. The introduction of CI/CD has not been flawless. For example, half of organisations doing so do not include any security testing elements, according to a 451 Research survey of 350 enterprise IT decision-makers in North America and Europe fielded in May 2018 \cite{clark}. This might be due to that the developers want to deliver the features faster then it should take. While speed is important, it is important to go as fast as possible but no faster. Going too fast can introduce errors and defects, which can be costly to fix \cite{clark}. 

To summarise that, the CI/CD pipeline is an essential DevOps practice that automates the software delivery process. It provides immediate feedback to developers, promotes efficiency and agility, and saves time and money. However, it can be complex and require a significant learning curve. CI/CD pipeline can encourage developers to go too fast before they are ready. Developers need to be aware of potential risks, such as security vulnerabilities and the need for human supervision in some tests. Ultimately, the CI/CD pipeline can help organizations deliver software faster and more reliably, but it requires a disciplined approach to implementation and management.

\section{Agile Methodology}
Agile software development methodology is becoming increasingly popular as it aims to address the insufficiency in traditional software development processes. With over 90\% of companies utilizing an agile approach for software development, it seeks to ensure a close link between the customer and developers to ensure that software meets market needs while striving for a more rapid release schedule \cite{hlrf}. Table~\ref{tab:swmethod} compares agile with traditional methods which proves why agile is preferable than traditional methods.

Agile methodology is characterized by its light nature, which allows for a quick response to change \cite{koch}. Survey studies by the Standish group have reported that less than 50\% of software development projects are successfully delivered within the set time, budget, and scope. Agile methods emphasize continuous feedback and improvement, leading to quicker responses and faster delivery \cite{dragos}. The methodology also ensures that each feature is "polished" repeatedly until it meets user requirements \cite{aaa}. Each release adds new functionalities, which improves the overall quality of the software. Agile also optimizes communication, favoring face-to-face communication, which enhances the overall quality of communication between the customer and developers\cite{koch}.

Although Agile methodology is effective in direct communication, it is considered weak in documentation, making it less suitable for long-term software maintenance \cite{aaa, koch}. The lack of emphasis on documentation can lead to different recollections of the same exchange, which can cause confusion and errors in the development process. Despite the benefits of achieving a more frequent cadence of software development, a bottleneck has emerged due to misalignment between the Operations function (Ops) and the Development function (Dev) \cite{hlrf}. This misalignment can cause long delays in software releases to customers, ultimately hindering the overall success of the project.

Overall, Agile is a suitable methodology for projects that demand flexibility, quick responses to change, and close collaboration between developers and customers. Conversely, long-term projects or maintenance is not recommended to use Agile as documentation is not prioritised which may cause problem in the long run. It is important to consider the nature of the software development project and choose the most suitable methodology accordingly. 

\section{DevOps Approach}
DevOps is the combination of development and operations practices that emphasize automation, collaboration, and monitoring throughout the software development lifecycle \cite{joakim}. DevOps is an extension of Agile methodology, incorporating operations practices to provide an end-to-end solution from design to monitoring, with a focus on automation and continuous feedback. DevOps emerged to address the deficiency that prevented development teams from delivering to operations in a faster and more frequent manner \cite{hlrf}. Many organizations have adopted DevOps practices to improve their software development processes as shown in various surveys and studies \cite{spj}.

One of the main advantages of DevOps is minimal manual intervention, allowing for quick and seamless deployment of applications \cite{spj}. By automating development, build, test, and deployment, DevOps can boost performance and efficiency, making the software development process faster, more efficient, and more reliable. The automation process also removes manual labour, freeing up team members to focus on other tasks \cite{joakim}. Moreover, DevOps emphasizes collaboration between development and operations teams, enabling faster delivery speed, reliability, and scalability. DevOps also enhances operations by configuring, deploying, and monitoring applications, leading to better utilization of resources and less time wasted \cite{os}. The continuous integration (CI) aspect of DevOps makes applications more robust and reliable. By continuously testing and integrating code, DevOps ensures that code is always production-ready. The automation aspect of DevOps also helps in identifying and fixing issues quickly, leading to improved software quality.

However, there are some challenges associated with implementing DevOps. One of the major challenges is that it is easy to lose track of the goal when working with DevOps implementations \cite{joakim}. DevOps is about delivering increased business value, not just about doing things faster. Therefore, it is important to keep track of the goal and measure the outcomes to ensure that the DevOps implementation is delivering value. Another challenge with DevOps is that it requires a significant cultural shift in the organization. The cultural shift can be difficult to achieve, as it requires all team members to work together and take ownership of the process. Additionally, the automation aspect of DevOps requires expertise in the use of automation tools, which can be a challenge for some team members. Finally, DevOps can be expensive to implement, requiring the use of expensive automation tools and a dedicated team to manage the automation process. The initial investment may be high, and it may take some time to realize the benefits of DevOps.

In conclusion, DevOps is a software development methodology that has become increasingly popular in recent years due to its ability to enhance automation, collaboration, and monitoring throughout the software development lifecycle. DevOps has several advantages, such as minimal manual intervention, faster deployment, and improved reliability. However, there are also challenges associated with DevOps, including the need for a cultural shift, expertise in automation tools, and high implementation costs. Organizations must weigh the pros and cons of DevOps and determine if it is the right methodology for their software development needs.

\section{Heroku}
Heroku is a platform as a service (PaaS) that changes the way web applications are deployed by providing an easy-to-use platform that allows developers to deploy, manage, and scale their applications \cite{mike}. In 2012, Heroku won the InfoWorld Technology of the Year award, cementing its position as a leading platform as a service provider with focus on infrastructure \cite{greengard, news}.

One of the significant advantages of Heroku is its flexibility \cite{greengard, anubhav, patrick}. Heroku supports a wide range of programming languages, including many modern open-source languages such as GO and Ruby. The list of supported languages is still growing, which makes it an attractive option for developers who want to work with the latest tools and technologies. Additionally, Heroku is highly scalable, and it scales transparently as traffic spikes, making it capable of serving applications with over 10,000 sustained requests per second today \cite{anubhav}.

Heroku's user-friendly interface and ease of use are also an advantage \cite{greengard}. Developers can easily connect to Heroku from Git, Github, and Docker, and the platform's adaptability means that it can adapt itself to the needs and requirements of the software. This makes it a popular choice for developers who want to focus on coding and development, rather than infrastructure management.

However, there are some limitations to using Heroku. One of the significant disadvantages is that heavy projects do not always run well on the platform \cite{greengard}. Heroku is designed to be easy to use and manage, which means that it may not be the best option for large, complex applications that require high levels of customization and control. Additionally, Heroku's pricing structure can be relatively expensive compared to other cloud platforms, especially for projects that require a lot of resources.

Overall, Heroku is a popular platform as a service that provides developers with a flexible, scalable, and user-friendly way to deploy and manage their web applications. Its focus on infrastructure and support for modern open-source languages makes it a popular choice for developers. However, its limitations, such as poor performance for heavy projects and a relatively expensive pricing structure, may make it unsuitable for certain projects. Developers should consider the specific requirements of their project before deciding to use Heroku as their deployment platform.