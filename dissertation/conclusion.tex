\chapter{Conclusion}
The significance of software development has increased significantly, and businesses heavily rely on applications to meet customer demands \cite{hf}. However, traditional development methods are no longer sufficient to meet the ever-increasing need for faster and more reliable software delivery \cite{sb}. Automation has emerged as a solution to tackle the problem of slow software development delivery. Continuous integration and continuous deployment (CI/CD) development practices have become increasingly popular for productivity and efficiency in recent years.

During this research, it became evident that Go, despite being initially considered easy to learn and use, can be quite challenging. Additionally, setting up CI/CD pipelines can be difficult, especially for those who are new to it or want to integrate an existing project. Different projects have to set up the pipeline in their way or according to different factors. However, the effort of setting up the pipelines is worth it, as it saves a lot of time when making changes to the code. CI/CD should be used more widely as it involves testing before deployment, which increases reliability.

The developer successfully incorporated a CI/CD pipeline into the project, meeting the objectives of the project. However, there is room for improvement in terms of planning and research in future projects. The use of tools like Jira can aid in better development planning and management, helping teams stay on track. Despite the need to recode the entire backend, the project was completed on time. To enhance the project's efficiency, it is crucial to consider the security of the database and take necessary measures to safeguard it against external threats. In summary, while the developer achieved the project's objectives, there is still room for improvement in terms of better planning and research for future projects, as well as enhancing the security of the database.

Overall, the adoption of continuous integration and continuous deployment (CI/CD) pipelines can greatly benefit organizations looking to streamline their software development processes \cite{kubinyi}. By automating the process of testing and deploying software, organizations can accelerate the development and delivery of software features, resulting in a shorter time-to-market, increased productivity, and a better overall customer experience \cite{kubinyi}. 

\section{Implications}
This paper contributed to the subject area by examine the effectiveness of CI/CD pipeline to an application. The adoption of continuous integration and continuous deployment pipelines can greatly improve the efficiency and effectiveness of software development processes. By automating the testing, deployment, and delivery of software, developers can reduce the risk of errors and ensure faster release cycles. This approach can also improve collaboration among development teams, operations, and other stakeholders, ultimately leading to better software quality and user satisfaction. More research or projects should be conducted to examine different aspects that will contribute to the CI/CD approach field so that developers and companies will be conscious of how it can affect software development processes.

\section{Limitations}
Despite the significant implications of this study, there were several limitations that must be acknowledged. First, the study was limited by the availability of data and resources. The developer did not have the final system written in GO for the backend due to this. Second, due to the time constraints of the project, the research focused on a limited number of CI/CD tools and practices. As a result of a short time frame, the study did not evaluate the long-term effectiveness of CI/CD pipelines, and additional research might need to determine the sustainability of these practices in the long run. Not only that, since the project team consisted of only one member, it may not have been comprehensive enough to include all potential or related sections and components.

\section{Recommendations}
In spite of the limitations of this study, the researcher anticipates further researchers can deal with the limitations of this study, hence, overcome them. Researchers should aim to replicate this study in different contexts to determine the generalisability of the findings. Future studies should explore the effectiveness of a wider range of CI/CD tools and practices to determine the most effective approaches. Researchers should investigate the long-term effectiveness of CI/CD pipelines and evaluate their sustainability over time. Organizations that are considering the adoption of CI/CD pipelines should conduct a thorough analysis of their development environment, structure, and tools to ensure that they have the necessary resources and expertise to effectively implement these practices. Furthermore, project teams should consist of more than one person to ensure quality, and scope needs to be planned properly with efficient research to avoid encountering problems in the middle of the development process.