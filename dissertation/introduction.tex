\chapter{Introduction}

As the digital transformation continues to accelerate, software development has become more crucial than ever before, with businesses relying heavily on applications to meet the demands of customers \cite{hf}. However, developing a decent software is a complex process which takes a long period of time and effort. With the ever-increasing need for faster and more reliable software delivery, traditional development methods are no longer sufficient \cite{dsmgm}. As software development processes can involve multiple stages, it takes a significant amount of time and effort to complete. Moreover, developers may be assigned with multiple projects simultaneously, which can result in further delays. As such, this can lead to customer dissatisfaction and lost opportunities \cite{sb}.

Automation has emerged as a solution to the issue of slow software development delivery. It improves software development's time and cost effectiveness by allowing software development teams to use resources more efficiently. Automation, for example, can aid in tasks requiring creative thinking or technical expertise, such as system integration, feature design, and debugging. Furthermore, the implementation of automation can decrease the duration needed to accomplish repetitive and mundane tasks, leading to a decrease in operational expenses. The use of automation, whether implemented partially or completely, can significantly improve software development processes \cite{saarenpaa2020creating}.

Continuous Integration and Continuous Deployment (CI/CD) development practices have gained widespread popularity in recent years due to their effectiveness and efficiency in software development. Despite its growing popularity, some may argue that setting up CI/CD can be a time-consuming and complex process \cite{sander}. Therefore, it is vital to conduct research on this topic to ensure that development teams can effectively implement and reap the benefits of CI/CD.

\section{Background Information}

Continuous Integration and Continuous Deployment have become crucial components for efficient and effective software delivery. According to a survey conducted by a software testing company, Mabl, about half of the 500 participants are already implementing Continuous Integration (CI), and a quarter plan to do so in the near future. Conversely, 190 out of 500 respondents use Continuous Deployment (CD) while another 150 respondents have plans to implement it \cite{clark}. These findings demonstrate the increasing recognition and adoption of CI/CD in software development, highlighting the need for developers to keep up with this trend in order to remain competitive in the industry.

According to Kubinyi, automating testing, deployment, and delivery processes by streamlining the CI/CD pipeline can reduce the risk of errors and ensure faster release cycles for development teams \cite{kubinyi}. These practices help developers automate the process of building, testing, and deploying their applications, allowing them to deliver software updates more quickly and efficiently. This approach can also improve collaboration among development teams, operations, and other stakeholders, ultimately leading to better software quality and user satisfaction \cite{bs}. Hence, CI/CD brings numerous advantages to software development teams and companies.

\section{Problem Statement}
According to multiple studies \cite{sb, saarenpaa2020creating, saz, chen, dm, phillips2015manager}, CI/CD pipelines enable the quick and dependable delivery of software changes, making it possible to implement new or amended product and service features within a shorter time frame. This significantly reduces the time required to implement ideas or fixes from weeks or months to a matter of days. This has shown that CI/CD pipelines have a positive impact on software development process.

However, it is true that setting up the pipelines can be difficult especially for those who are new to it or want to integrate an existing project \cite{sander}. Hence, it is crucial to have the be prepared to acknowledge the challenges that come during implementation as it requires a deep understanding of the development environment, structure and tools involved.

In response to the problem, the purpose of this research paper is to study and examine the effectiveness of CI/CD pipeline while designing a web application to test and deploy.

\section{Significance of the Study}
The outcome of this project can help organisations to benefit greatly from the adoption of CI/CD pipelines. Organisations can accelerate the development and delivery of software features by automating the process of testing and deploying software. This can result in a shorter time-to-market, increased productivity, and a better overall for customers experience \cite{hf}.

Thus, clients or customers can receive more and faster feedback, as well as higher quality software as developers can detect and fix bugs earlier in the development cycle With CI/CD pipelines. Customers can also benefit from a more responsive and dynamic product with the ability to deploy new features and updates more frequently. \cite{chen, leppanenetal}.

Finally, this research can help the researcher and other learners to gain a deeper understanding of CI/CD principles. Teams can achieve a more efficient workflow, better collaboration, and higher code quality by automating the software development process, all of which are critical for project success \cite{sander}. In this context, for instance, a website has been developed and implemented with a CI/CD pipeline to better understand everything from the ground up.

\section{Objectives}
\begin{enumerate}
  \item To demonstrate an understanding of principles and practices of continuous integration and deployment for better efficiency.
  \item To enhance designing and developing a scalable web application using modern front-end frameworks and back-end technologies.
  \item To implement best practices for software development, such as testing and documentation.
  \item To develop skills in project management, including project planning and tracking.
\end{enumerate}

\section{Project Repository Overview and Key Components}
This dissertation presents the design and implementation of a CI/CD pipeline using GitHub Actions, while also developing a simple student grade management website using MERN stack with basic CRUD functionalities. The website is built using React.js as the front-end and MongoDB as the database, with Node.js and Express.js serving as the back-end server.

A functioning student grade management website that incorporates basic CRUD functionalities and demonstrates continuous integration and deployment principles and practices has been designed. The code for this project is hosted on GitHub, and can be found \href{https://github.com/gabhang/final-year-project}{here}. The website is built using React.js as the front-end and MongoDB as the database. Initially, the back-end server was developed in Go language, but it was later changed to Node.js and Express.js due to deployment compatibility and issues. Next, Jest and Supertest were used to write tests for testing, and Heroku was used for deployment. Lastly, Jira, a project management software, is utilised to plan and track project progress.

The student grade system deployed from this repository can be accessed \href{https://student-grade-system.herokuapp.com/}{here} and the following directory layout shows the hierarchy of directories and files for the project with description:\newline

\dirtree{%
.1 student-grade-system.
.2 .github/workflows.
.3 checks.yml.
.2 BACKEND.
.3 server.js.
.2 dissertation.
.2 public.
.2 src.
.3 components.
.4 create.SG.js.
.4 listings.js.
.4 updateSG.js.
.3 App.js.
.3 App.css.
.2 test.
.3 crud.test.js.
.2 .gitignore.
.2 Procfile.
.2 README.md.
.2 package-lock.json.
.2 package.json.
}

\begin{itemize}
\item .github/workflows/checks.yml: This file sets up GitHub Actions to run checks on each push request to ensure that the code passes all tests and deploys automatically to Heroku.
\item BACKEND/server.js: This file contains the back-end code for the project, which handles CRUD API requests and database interactions.
\item dissertation: This directory contains the author's dissertation in LaTeX format.
\item public: This directory contains public files, such as images and static HTML files.
\item src: This directory contains the main React source code for the project, including the components sub-directory that contains different pages of the website. The App.js file provides a navigation bar for every page, while the App.css file manages the CSS styles for the website.
\item tests/crud.test.js: This file contains tests for the project, testing the Create, Read, Update, and Delete (CRUD) functionality.
\item .gitignore: This file specifies files and directories to be ignored by Git when committing changes. For this project, the node\_modules and build folders are ignored, for instance.
\item Procfile: This file is used by Heroku to specify the commands to run when the app is deployed. In this project, this file is used to start the server.
\item README.md: This file provides an overview of the project.
\item package-lock.json: This file specifies version numbers for dependencies to ensure consistency across installations.
\item package.json: This file specifies the project's dependencies and scripts for running the app.
\end{itemize}